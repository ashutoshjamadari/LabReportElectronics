\documentclass[onecolumn,12pt]{article}
%\twocolumn
\usepackage{amsmath}
\usepackage{graphicx}

%\usepackage{cite}

\begin{document}

\title{A sample LaTeX file}
\author{ Mona)}
\date{ prepared on \today}

\maketitle
%\today

\begin{abstract}
you can write the abstract of your work here ... 
\end{abstract}



\listoftables
\listoffigures

\newpage

\section{Introduction}
This is sample section.\newline this is how we cite\cite{LatexLearn}
\subsubsection{Subsubsection}
This is sample subsubsection.
\newline Refer to the Math equation \ref{eqn:yf} below
\begin{equation}
%\tiny
{A_3 = \frac{K_2 K_3 A}{64(K_3 V_{cmi}+K_4)(2K_1\sqrt{K_3 V_{cmi}+K_4}-K_2 K_3)}}
\label{eqn:yf}
\end{equation}




\begin{table}[h]
	\centering
	\caption{ An example table}
	\label{table:bwg}
\begin{tabular}{|c|c|c|c|}
	\hline
	Process & Bandwidth & Gain & SR\\
	corner & (GHz) & (dB) & (v/ns)\\
	\hline
	ss & 1 & 50 & 10 \\
	\hline
	tt & 1.2 & 45 & 20 \\
	\hline
	ff & 1.4 & 55 & 0.1 \\
	\hline
\end{tabular}

\end{table}




\begin{thebibliography}{9}

	\bibitem{LatexLearn} 
	ShareLatex Documentation
	\\\texttt{https://www.sharelatex.com/learn}
	
	\bibitem{UserGuide} 
Leslie Lamport 
\textit{LATEX A document preparation system user's guide and Reference Manual}. 
Pearson, Tenth Impressions, 2015.


	
\end{thebibliography}
\end{document}
